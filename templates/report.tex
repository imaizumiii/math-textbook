\documentclass[a4paper]{article}
\usepackage{amsmath}
\usepackage{amsfonts}
\usepackage{amssymb}
\usepackage[utf8]{inputenc}
\usepackage[T1]{fontenc}
\usepackage{graphicx}
\usepackage{hyperref}
% 日本語対応(pdflatexの場合)
\usepackage{CJKutf8}

\begin{document}
\begin{CJK}{UTF8}{min}
\begin{center}
{\Large\bfseries {{- title -}} }\\[1em]
{{- author -}} \\[0.5em]
{{- date -}}\today
\end{center}
\vspace{2em}


\begin{abstract}
{{ abstract }}
\end{abstract}


\section{計算結果}
Pythonで計算された円周率(の一部)を数式として表示します。

\[
    \pi \approx \mathbf{3.141593}
\]

この文書は、Pythonの実行によって完全に自動生成されました。

\end{CJK}
\end{document}
