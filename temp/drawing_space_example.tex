\documentclass[a4paper]{article}
\usepackage{fontspec}
\usepackage{xeCJK}
\usepackage{amsmath}
\usepackage{amsfonts}
\usepackage{amssymb}
\usepackage{graphicx}
\usepackage{hyperref}
\usepackage{tcolorbox}

% フォント設定
\setCJKmainfont{Noto Sans JP}[Path=fonts/, UprightFont=NotoSansJP-Regular.ttf, BoldFont=NotoSansJP-Bold.ttf]

\usepackage[top=2cm,bottom=2cm,left=2cm,right=2cm]{geometry}
\usepackage{setspace}
\setstretch{1.5}

\begin{document}
\begin{center}
{\Large\bfseries 手書き用余白の例 }\\[1em]
PDF Generator \\[0.5em]
\today
\end{center}
\vspace{2em}

\section{通常のテキスト}

この段落は通常通り、ページ全体の幅を使用します。


複数の段落を追加することもできます。



\section{手書き用余白を持つ部分}

以下の部分は、右側に余白が確保されています。


\begin{minipage}[t]{0.7\textwidth}
この段落は幅が制限され、右側に5cmの余白があります。


この余白部分に、後から手書きでグラフや図を描くことができます。


\[
    f(x) = x^2 + 2x + 1
\]
数式も同じ幅に制限されます。
\end{minipage}
\hspace{5cm}
\par
\vspace{0.5em}

通常のテキストに戻ります。



\section{異なる幅の余白}

\begin{minipage}[t]{0.6\textwidth}
より狭い幅(60%)と、より広い余白(6cm)の例です。


\begin{tcolorbox}[title={例題}]
この問題の解答を右側の余白に図示してください。
\end{tcolorbox}

\end{minipage}
\hspace{6cm}
\par
\vspace{0.5em}


\section{セクション内での使用例}

セクション内でも使用できます。
\begin{minipage}[t]{0.7\textwidth}
デフォルトの幅(70%)と、カスタム余白(4cm)の例です。


\[
    \int_0^1 x^2 dx = \frac{1}{3}
\]
\end{minipage}
\hspace{4cm}
\par
\vspace{0.5em}


\end{document}
