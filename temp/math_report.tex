\documentclass[a4paper]{article}
\usepackage{fontspec}
\usepackage{xeCJK}
\usepackage{amsmath}
\usepackage{amsfonts}
\usepackage{amssymb}
\usepackage{graphicx}
\usepackage{hyperref}
\usepackage{tcolorbox}

% フォント設定
\setCJKmainfont{Noto Sans JP}[Path=fonts/, UprightFont=NotoSansJP-Regular.ttf]

\usepackage[top=2cm,bottom=2cm,left=2cm,right=2cm]{geometry}

\begin{document}
\begin{center}
{\Large\bfseries 数学レポート }\\[1em]
あなたの名前 \\[0.5em]
2024年1月1日
\end{center}
\vspace{2em}

\begin{abstract}
このレポートでは、PythonとLaTeXの連携について説明します。
\end{abstract}

\section{はじめに}

このレポートでは、以下の内容について説明します。
\begin{enumerate}
    \item PythonによるLaTeX生成
    \item 図の挿入方法
    \item テキストボックスの使用方法
\end{enumerate}


\section{計算結果}

Pythonで計算された円周率を数式として表示します。
\[
    \pi \approx \mathbf{3.141593}
\]

\section{補足説明}

\begin{tcolorbox}[colback=yellow!5!white, colframe=yellow!75!black, title={注意}]
この結果は実験的に確認されました。
\end{tcolorbox}

\begin{tcolorbox}[colback=red!5!white, colframe=red!75!black, title={警告}]
数値は近似値です。
\end{tcolorbox}

\begin{tcolorbox}[colback=blue!5!white, colframe=blue!75!black, title={情報}]
詳細は参考文献を参照してください。
\end{tcolorbox}


\section{いろいろ使ってみる}

\begin{align*}
    a &= b + c \\
    &= y^2 + 1 \\
    &= mc^2
\end{align*}

テキスト内に数式を書いています。$y = ax + b$

\end{document}
