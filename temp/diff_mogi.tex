\documentclass[a4paper]{article}
\usepackage{fontspec}
\usepackage{xeCJK}
\usepackage{amsmath}
\usepackage{amsfonts}
\usepackage{amssymb}
\usepackage{graphicx}
\usepackage{hyperref}
\usepackage{tcolorbox}
\usepackage{xcolor}
\usepackage{multicol}

% フォント設定
\setCJKmainfont{Noto Sans JP}[Path=fonts/, UprightFont=NotoSansJP-Regular.ttf, BoldFont=NotoSansJP-Bold.ttf]

\usepackage[top=2cm,bottom=2cm,left=2cm,right=2cm]{geometry}
\usepackage{setspace}
\setstretch{1.8}

\begin{document}
\section*{Theme: 微分積分とは?}

\noindent みなさん、\textbf{”微分積分”}って聞いたことありますか? 「難しそう」という感想を持つ人がほとんどだと思いますが、実はすごくシンプルで便利なツールなんですよ。


\noindent まずは、復習から。\textbf{”変化の割合”}という言葉はきいたことあると思います。$\text{変化の割合}=\frac{\text{$y$の増加量}}{\text{$x$の増加量}}$とかいうやつです。具体例を用いながら思い出していきましょう。


\vspace{0.0em}
\begin{center}
*\hspace{10em} *\hspace{10em} *
\end{center}
\vspace{0.0em}

\noindent
\makebox[\textwidth][l]{%
\begin{minipage}[t]{0.65\textwidth}
\noindent $y=x^2$というグラフを考えてみましょう。$x=1$から$x=2$に変化したときの変化の割合は、次のような式から求められます。


\noindent \begin{center}$\dfrac{\text{$y$の増加量}}{\text{$x$の増加量}} =\dfrac{f(2) - f(1)}{2 - 1} =\dfrac{2^2 - 1^2}{2 - 1} = 3$\end{center}
\noindent さて、ここからは文字にしていきます。文字にしたとたんにわからない人が急増するのでお気を付けを。
\noindent \begin{center}$\dfrac{\text{$y$の増加量}}{\text{$x$の増加量}} =\dfrac{f(x+h) - f(x)}{(x+h) - x} =\dfrac{f(x+h) - f(x)}{h}$\end{center}
\noindent 具体例では\textbf{「} $\boldsymbol{1 \rightarrow 2}$ \textbf{」}という変化でしたが、これを\textbf{「} $\boldsymbol{x \rightarrow x+h}$ \textbf{」}という変化にしてみました。グラフからよく確認しておいてください。
\end{minipage}
\hspace{20cm}
}%
\par
\vspace{1em}

\noindent このタイミングで微分係数を求める公式(定義)を紹介することにします。
\begin{tcolorbox}[before upper={\setlength{\abovedisplayskip}{5pt}\setlength{\belowdisplayskip}{5pt}\setlength{\abovedisplayshortskip}{0pt}\setlength{\belowdisplayshortskip}{0pt}}, title={微分係数の求め方(定義)}]
関数$f(x)$の微分係数$f'(x)$は以下のように定義される。\[ f'(x) = \lim_{h \to 0} \frac{f(x+h) - f(x)}{h} \]
\end{tcolorbox}

\noindent {\centering\textbf{さっき作った式にすごく似ていることがわかりますね?}\par}
\noindent さっきの図をでは、\textbf{幅を} $\boldsymbol{h}$としていたので、「\textbf{その幅をごくごく小さくすれば、瞬間の変化率がわかるんじゃね?」}という発想になります。実際にさっきの例で計算してみましょうか。
\begin{align*}
    \begin{aligned}
    f'(x)      & = \lim_{h \to 0} \frac{f(x+h) - f(x)}{h} \\
               & = \lim_{h \to 0} \frac{(x+h)^2 - x^2}{h} \\
               & = \lim_{h \to 0} \frac{x^2 + 2xh + h^2 - x^2}{h} \\
               & = \lim_{h \to 0} \frac{2xh + h^2}{h} \\
               & = \lim_{h \to 0} (2x + h) \\
               & = 2x
    \end{aligned}
\end{align*}

\noindent よって、$\boldsymbol{f'(x) = 2x}$となります。実際に、「$f(x) = x^2$の\textbf{導関数}を求めよ」と言われたら、これが答えになります。「$x=2$における$f(x) = x^2$の\textbf{微分係数}を求めよ」とか言われたら、$x=2$を代入して、$f'(2) = 4$が答えになります。少しだけ単語が複雑ですが、問題を解くときにはあまり困らないので、あまり気にしないでおくことにしましょう。\\
\noindent \textbf{これがとても便利なんですよ。}今回はこれで終わりですが、この分野は物理や数学以外にも、経済学や社会学でもよく使われるんですよ。では、少しだけ教科書の問題を解いて終わることとしましょう。


\par\noindent
\makebox[\textwidth][s]{\textcolor{gray}{\leaders\hrule height 5pt \hfill}\quad\textbf{練習4, 5(教科書P.195,199)}\quad\textcolor{gray}{\leaders\hrule height 5pt \hfill}}
\par

\noindent
\textbf{練習4}\quad 関数$f(x) = x^3+2$の$x=1$における微分係数$f'(1)$を求めよ。
\par
\vspace{0.5em}

\noindent
\textbf{練習5}\quad 導関数の定義に従って、次の関数の導関数を求めよ。
\vspace{-1.5em}
\begin{multicols}{2}
\begin{enumerate}
  \renewcommand{\labelenumi}{(\arabic{enumi})}
  \item $f(x) = 3x^2$
  \item $f(x) = -x^2$
\end{enumerate}
\end{multicols}
\par
\vspace{0.5em}


\end{document}
